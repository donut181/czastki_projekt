{
	\centering
	{\large\bfseries Sprawozdanie z projektu z przedmiotu \\
	Cząstki Elementarne i Ich Oddziaływania.\\}
	\vspace{1em}
	{
	Jakub Ahaddad\\

	Marcin Polok\\
	}
	\vspace{3em}
}

\section{Wstęp}
Celem projektu jest wyznaczenie masy i czasu życia jednego z 3 dostępnych mezonów przedstawionych w tabeli \ref{tab_mesons}. Podstawą do wyznaczenia tych parametrów
są rzeczywiste dane zebrane w spektrometrze LHCb w 2016r. Część praktyczną projektu przeprowadzono za pomocą
oprogramowania ROOT, pisząc własny skrypt.

\begin{table}[H]
\centering
\caption{Parametry możliwych do wyboru mezonów\cite{database}}
\label{tab_mesons}
\begin{tabular}{ccccccc}

Cząstka		&masa [MeV]	&spin	&ładunek&J$^\text{PC}$	&Średni czas życia [s]		&\textcolor{red}{Średnia droga???} 	\\ \hline

D$^{*+}$	&2010.26±0.05	&0	&+1	&1$^{-+}$	&(1040±7)$\cdot10^{-10}$	&		\\
D$^0$		&1863.84±0.05	&0	&0	&0$^{-+}$	&(410.1±1.5)$\cdot10^{-10}$	&		\\
K$^0_\text{S}$	&497.611±0.013	&0	&0	&0$^{-+}$	&(0.8954±0.0004)$\cdot10^{-10}$	&		\\

\end{tabular}
\end{table}
