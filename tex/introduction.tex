{
	\centering
	{\large\bfseries Sprawozdanie z projektu z przedmiotu \\
	Cząstki Elementarne i Ich Oddziaływania.\\}
	\vspace{1em}
	{
	Jakub Ahaddad\\

	Marcin Polok\\
	}
	\vspace{3em}
}

\section{Wstęp}
Celem projektu jest wyznaczenie masy i czasu życia jednego z 3 dostępnych mezonów przedstawionych w tabeli \ref{tab_mesons}. Podstawą do wyznaczenia tych parametrów
są rzeczywiste dane zebrane w spektrometrze LHCb w 2016r. Część praktyczną projektu przeprowadzono za pomocą
oprogramowania ROOT, pisząc własny skrypt.

\begin{table}[H]
\centering
\caption{Parametry możliwych do wyboru mezonów\cite{database:K}\cite{database:D0}\cite{database:D*}}
\label{tab_mesons}
\begin{tabular}{ccccccc}

Cząstka		&masa, $\mega\electronvolt$	&spin	&ładunek&J$^\text{PC}$	&Średni czas życia, $\second$		&\textcolor{red}{Średnia droga???} 	\\ \hline

D$^{*+}$	&$2010.26 \pm 0.05$	&0	&+1	&1$^{-+}$	&$(1040 \pm 7)\cdot 10^{-10}$	&		\\
D$^0$		&$1863.84 \pm 0.05$	&0	&0	&0$^{-+}$	&$(410.1 \pm 1.5)\cdot 10^{-10}$	&		\\
K$^0_\text{S}$	&$497.611 \pm 0.013$	&0	&0	&0$^{-+}$	&$(0.8954 \pm 0.0004)\cdot 10^{-10}$	&		\\

\end{tabular}
\end{table}
